\makeatletter
%%设计封面%%
\newcommand{\ifetdef}{
\maxsizebox{\paperwidth}{!}{\ifdefvoid{\@englishtitle}{}{\@englishtitle}}
}
\newcommand{\makecover}{
\tikzset{coverup node/.style={text=ColorA,anchor=north west,inner sep=0mm},
coverdown node/.style={text=white,anchor=west,inner sep=0mm},
opacity node/.style={anchor=north west,font=\FontSizeSet{100pt}{120pt}\bfseries\sffamily,text=ColorB!25}
}
\begin{titlepage}
\begin{tikzpicture}[remember picture,overlay]
\coordinate(A)at(current page.north west);
\coordinate(B)at(current page.north east);
\coordinate(C)at(current page.south east);
\coordinate(D)at(current page.south west);
\coordinate(E)at([yshift=5.25cm]current page.west);
\coordinate(F)at([yshift=5.25cm]current page.east);
\coordinate(G)at($0.5*(E)+0.5*(C)$);
%%背景1%%
\draw[ColorB!5,fill=ColorB!5,sharp corners](D)rectangle(B);
%%半透明文字%%
\node[opacity node,text opacity=1.0](O_1)at(A){\ifetdef};
\foreach \x[evaluate=\x as \botpoint using int(\x-1),
evaluate=\x as \opa using 1.2-0.2*\x] in {2,...,10}{
\node[opacity node,text opacity=\opa](O_\x)at(O_\botpoint.south west){\ifetdef};}
%%背景2%%
\draw[ColorA!80,fill=ColorA!80](E)rectangle(C);
%%封面图%%
\node[inner sep=0mm,anchor=center](CI)at(G){\ifdefvoid{\@coverimage}{}{\includegraphics[width=\paperwidth,height=\paperheight/2+5.25cm]{\@coverimage}}};
%%标题%%
\node[font=\FontSizeSet{55pt}{66pt}\bfseries\rmfamily,coverup node](T)at(current page text area.north west){\@title};
%%英文标题%%
\node[font=\FontSizeSet{20pt}{24pt}\bfseries\sffamily,coverup node](ET)at([yshift=-0.5cm]T.south west){\ifdefvoid{\@englishtitle}{}{\@englishtitle}};
%%副标题%%
\node[font=\FontSizeSet{30pt}{36pt}\bfseries\rmfamily,coverup node](ST)at([yshift=-0.5cm]ET.south west){\ifdefvoid{\@subtitle}{}{\@subtitle}};
%%英文副标题%%
\node[font=\FontSizeSet{15pt}{18pt}\bfseries\sffamily,coverup node](EST)at([yshift=-0.5cm]ST.south west){\ifdefvoid{\@englishsubtitle}{}{\@englishsubtitle}};
%%作者%%
\node[font=\FontSizeSet{15pt}{18pt}\bfseries\rmfamily,coverdown node](A)at([yshift=4.0cm]current page text area.west){主编\quad\@author};
%%出版社%%
\node[coverdown node](PL)at(current page text area.south west){\ifdefvoid{\@presslogo}{}{\includegraphics[width=0.75cm]{\@presslogo}}};
\node[font=\FontSizeSet{15pt}{18pt}\bfseries\rmfamily,coverdown node](PN)at([xshift=0.25cm]PL.east){\ifdefvoid{\@pressname}{}{\@pressname}};
\end{tikzpicture}
\end{titlepage}}

%%设计封底%%
\newcommand{\ifestdef}{
\maxsizebox{\paperwidth}{!}{\ifdefvoid{\@englishsubtitle}{}{\@englishsubtitle}}
}
\NewDocumentCommand{\makeback}{ +O{} o }{
\tikzset{backup node/.style={text=ColorA,anchor=north east,inner sep=0mm,text width=\textwidth},
backdown node/.style={text=white,anchor=west,inner sep=0mm},
opacity node/.style={anchor=north east,font=\FontSizeSet{100pt}{120pt}\bfseries\sffamily,text=ColorB!25}}
\begin{titlepage}
\begin{tikzpicture}[remember picture,overlay]
\coordinate(A)at(current page.north west);
\coordinate(B)at(current page.north east);
\coordinate(C)at(current page.south east);
\coordinate(D)at(current page.south west);
\coordinate(E)at([yshift=5.25cm]current page.west);
\coordinate(F)at([yshift=5.25cm]current page.east);
\coordinate(G)at($0.5*(E)+0.5*(C)$);
%%背景1%%
\draw[ColorB!5,fill=ColorB!5,sharp corners](D)rectangle(B);
%%半透明文字%%
\node[opacity node,text opacity=1.0](O_1)at(B){\ifestdef};
\foreach \x[evaluate=\x as \botpoint using int(\x-1),
evaluate=\x as \opa using 1.2-0.2*\x] in {2,...,10}{
\node[opacity node,text opacity=\opa](O_\x)at(O_\botpoint.south east){\ifestdef};}
%%背景2%%
\draw[ColorA!80,fill=ColorA!80](E)rectangle(C);
%%封底图%%
\node[inner sep=0mm,anchor=center](BI)at(G){\ifdefvoid{\@backimage}{}{\scalebox{1}[1]{\includegraphics[width=\paperwidth,height=\paperheight/2+5.25cm]{\@backimage}}}};
%%条形码%%
\IfValueTF{#2}{
\node[draw=white,fill=white,inner xsep=0mm,anchor=south east](BC)at([yshift=-0.5cm]current page text area.south east){\includegraphics[width=5.0cm]{#2}};}{\relax}
%%说明文字%%
\node[font=\ChapterSayingFont,backup node](BT)at(current page text area.north east){\hspace*{2.0em}#1};
%%书籍系列%%
\node[font=\FontSizeSet{15pt}{18pt}\bfseries\rmfamily,backdown node](S)at([yshift=4.0cm]current page text area.west){\ifdefvoid{\@series}{}{\@series}};
%%书籍系列%%
\node[font=\FontSizeSet{15pt}{18pt}\bfseries\rmfamily,backdown node](V)at([yshift=-0.5cm]S.south west){\ifdefvoid{\@version}{}{第{\@version}版}};
\end{tikzpicture}
\end{titlepage}}

%%设计书脊%%
\NewDocumentCommand{\makespine}{O{1.0cm}}{
\newlength{\SpineWidth}
\setlength{\SpineWidth}{#1}
\tikzset{spine title/.style={font=\FontSizeSet{20pt}{24pt}\bfseries\rmfamily,text=white,text width=1.0cm,align=center},
spine subtitle/.style={font=\FontSizeSet{15pt}{18pt}\bfseries\rmfamily,text=white,text width=1.0cm,align=center},
spine englishtitle/.style={font=\FontSizeSet{20pt}{24pt}\bfseries\sffamily,text=ColorA,inner sep=0mm}}
\begin{titlepage}
\begin{tikzpicture}[remember picture,overlay]
\coordinate(A)at(current page.north);
\coordinate(B)at(current page.south);
\coordinate(C)at([yshift=5.25cm]current page.center);
%%背景%%
\draw[ColorB!5,fill=ColorB!5]
([xshift=-\SpineWidth/2]A)rectangle([xshift=\SpineWidth/2]B);
\draw[ColorA!80,fill=ColorA!80]
([xshift=-\SpineWidth/2]C)rectangle([xshift=\SpineWidth/2]B);
%%书脊英文标题%%
\node[spine englishtitle,anchor=north](ET)at([yshift=-0.5cm]A){\ifdefvoid{\@englishtitle}{}{\rotatebox[origin=c]{-90}{\maxsizebox{\paperheight/2-6.0cm}{\SpineWidth}{\@englishtitle}}}};
%%书脊标题%%
\node[spine title,anchor=north](T)at([yshift=-0.5cm]C){\@title};
%%书脊出版社名称%%
\node[spine subtitle,anchor=south](PN)at([yshift=0.5cm]B){\ifdefvoid{\@pressname}{}{\@pressname}};
%%书脊出版社logo%%
\node[spine subtitle,anchor=south](PL)at([yshift=0.25cm]PN.north){\ifdefvoid{\@presslogo}{}{\includegraphics[width=0.8cm]{\@presslogo}}};
\end{tikzpicture}
\end{titlepage}}
\makeatother