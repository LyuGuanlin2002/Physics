\documentclass[color=red,openany]{textbook-cn}%,splitbib
\graphicspath{{./figure/}{./figures/}{./image/}{./images/}{./graphics/}{./graphic/}{./pictures/}{./picture/}}%提供多种图片路径
\input{./styles/Cover.tex}
\input{./styles/Logo.tex}
\usepackage{zhlipsum,lipsum}%乱数假文(测试文字)
\usepackage{physics}%物理符号包
\usepackage{siunitx}
\usetikzlibrary{patterns,3d}

%%设置符号表样式%%
\nomensetup{columnsep=2.0em,columns=3}
\makenomenclature

%%设置索引样式%%
\indexsetup{}
\makeindex[name=noun,title=名词索引,columns=3,columnsep=2.0em,options=-s indexstyle.ist]
\makeindex[name=video,title=视频索引,columnsep=2.0em,options=-s indexstyle.ist]

%%设置参考文献样式%%
\bibsetup{intoc}

%%定义罗马数字%%
\makeatletter
\newcommand{\rmnum}[1]{\romannumeral #1}
\newcommand{\Rmnum}[1]{\expandafter\@slowromancap\romannumeral #1@}
\makeatother

%%设置书籍信息%%
\series{物理学系列教程}
\version{三}
\title{物理学基础讲义}
\englishtitle{Fundamental of Physics}
\subtitle{经典力学}
\englishsubtitle{Classical Mechanics}
\author{RQCEN}
\pressname{BILIBILI教育出版社}
\presslogo{gj.pdf}
\coverimage{coverimage.pdf}%
\backimage{coverimage.pdf}%

\newcommand{\indexnoun}[1]{\index[noun]{#1}}
\newcommand{\indexvideo}[1]{\index[video]{#1}}

\newcommand{\zhindexnoun}[2]{\zhindex[noun]{#1}{#2}}
\newcommand{\zhindexvideo}[2]{\zhindex[video]{#1}{#2}}


\newcommand{\inputplus}{\raisebox{-0.5em}{\Logo[0.07]{physics}{PLUS}{\faPlus}}~~~}
\newcommand{\inputlook}{\raisebox{-0.5em}{\Logo[0.07]{physics}{LOOK}{\faPlay}}~~~}
\newcommand{\inputlisten}{\raisebox{-0.5em}{\Logo[0.07]{physics}{LISTEN}{\faVolumeDown}}~~~}
\newcommand{\inputread}{\raisebox{-0.5em}{\Logo[0.07]{physics}{READ}{\faStickyNote}}~~~}


%%选择题的4个选项,根据选项内容长度自动排版%%
\colorlet{ListColor}{ColorA}
\newlength{\la}\newlength{\lb}\newlength{\lc}\newlength{\ld}
\newlength{\lhalf}\newlength{\lquarter}\newlength{\lmax}
\newcommand{\choice}[4]{
\settowidth{\la}{A.~#1~~~}\settowidth{\lb}{B.~#2~~~}
\settowidth{\lc}{C.~#3~~~}\settowidth{\ld}{D.~#4~~~}
\ifthenelse{\lengthtest{\la>\lb}}
{\setlength{\lmax}{\la}}{\setlength{\lmax}{\lb}}
\ifthenelse{\lengthtest{\lmax<\lc}}
{\setlength{\lmax}{\lc}}{\relax}
\ifthenelse{\lengthtest{\lmax<\ld}}
{\setlength{\lmax}{\ld}}{\relax}
\setlength{\lhalf}{0.5\linewidth}
\setlength{\lquarter}{0.25\linewidth}
\ifthenelse{\lengthtest{\lmax>\lhalf}}
{\begin{hlist}[pre skip=0pt,item skip=0pt,item offset={1.5em},label={\textbf{\color{ListColor}{\Alpha{hlisti}.}}},pre label={}]1
\hitem #1
\hitem #2
\hitem #3
\hitem #4
\end{hlist}}
{\ifthenelse{\lengthtest{\lmax>\lquarter}}
{\begin{hlist}[pre skip=0pt,item skip=0pt,item offset={1.5em},label={\textbf{\color{ListColor}{\Alpha{hlisti}.}}},pre label={}]2
\hitem #1
\hitem #2
\hitem #3
\hitem #4
\end{hlist}}
{\begin{hlist}[\parskip=0pt,pre skip=0pt,item skip=0pt,item offset={1.5em}, label={\textbf{\color{ListColor}{\Alpha{hlisti}.}}},pre label={}]4
\hitem #1
\hitem #2
\hitem #3
\hitem #4
\end{hlist}}
}}

%空心字母
\newcommand{\E}{\mathbb{E}}
\renewcommand{\Pr}{\mathbb{P}}
\newcommand{\EP}{\mathbb{E}^{\mathbb{P}}}
\newcommand{\EQ}{\mathbb{E}^{\mathbb{Q}}}
\newcommand{\dif}{\,{\rm d}}
\newcommand{\Var}{{\rm Var}}
\newcommand{\Cov}{{\rm Cov}}

%旧式正切余切函数名 tan cot tanh coth arctan arccot
\newcommand{\sh}{\operatorname{sh}}
\newcommand{\ch}{\operatorname{ch}}
\renewcommand{\th}{\operatorname{th}}
\newcommand{\tg}{\operatorname{tg}}
\newcommand{\ctg}{\operatorname{ctg}}
\newcommand{\tgh}{\operatorname{tgh}}
\newcommand{\ctgh}{\operatorname{ctgh}}
\newcommand{\arctg}{\operatorname{arctg}}
\newcommand{\arcctg}{\operatorname{arcctg}}

%龟壳括号,又称六角括号
\newcommand{\lbr}{\ensuremath{\lbrbrak}}
\newcommand{\rbr}{\ensuremath{\rbrbrak}}
\newcommand{\lhbrak}{\ensuremath{\lbrbrak}}
\newcommand{\rhbrak}{\ensuremath{\rbrbrak}}

%表格对齐设置
\newcolumntype{M}{>{\centering\arraybackslash}X}
\newcolumntype{Z}{>{\raggedright\arraybackslash}X}
\newcolumntype{Y}{>{\raggedleft\arraybackslash}X}

%%%%%%%%%%%%%%%%%%%%%%%%%%%%%%%%%%%%%%%%%%%%%%%%%%%%%%%%%%%%%%%%%%%%%%%%%%%%%%%%%%%
\begin{document}
\MainFont

\makecover
\maketitle

\frontmatter
\part{力和运动$\omega$}
\part{力和运动力和运动力和运动力和运动力和运动力和运动力和运动力和运动力和运动力和运动}
% 内容提要
\include{body/F03-Abstract}
% 序言
\include{body/F04-Foreword}

\tableofcontents
%
\mainmatter

\part{力和运动}
\part*{力和运动}
%% 第一章
%\begin{Test}
\chaptersaying{文王拘而演周易,仲尼厄而作春秋。}
\chapter{测量}





\section{测量物体,包含长度测量}

\begin{Point*}
	学习完这一单元,你应当能够: 识别国际单位制(SI)中的基本量; 正确称呼国际单位制中最常用的词头; 利用链环变换法改变单位(这里是对于长度、面积和体积); 说明米是如何用真空中光速来定义的。
\end{Point*}


\begin{Case*}
物理学是以物理量的测量为基础的。一些物理量被选作基本物理量; 每一种基本物理量都已经通过一种标准来定义并给定测量的单位。另一些物理量是用这些基本物理量和它们的标准及单位来定义。

本书主要用的单位制是国际单位制(SI)。基本物理量的标准必须都是可以得到并且是不变的,这些基本量的标准都已经通过国际协议确立了。这些标准用在所有的物理测量中,基本量和从这些基本量导出的量都是如此。

单位的变换可以通过链环变换法来完成,在这种变换法中,将原来的数据连续乘以统一协调的变换因子,单位要像代数量一样处理,直到留下所要求的单位。

米定义为在精确规定的时间间隔内光通过的路程。
\end{Case*}



\begin{Paracol}
	\subsection{什么是物理学?}
	
	科学和工程学建立在测量和比较的基础上。因此,我们就需要关于如何测量和比较物体的规则,并且我们还需要通过实验来建立这种测量和比较的单位。物理学(以及工程学)的一个目的就是设计和实施这些实验。
	
	例如,物理学家们尽力开发极其准确的时钟从而使任何时间或时间间隔都可以精确地确定和比较。你们可能要问,这样的准确度是否真正需要,或者是否值得花这么大的精力。这里给出一个是否值得的例子:如果没有极其准确的时钟的话,那么现在全世界范围的航行必不可少的全球定位系统(GPS)就会变得毫无意义。
	
	\subsection{测量物体}
	
	我们通过学习怎样测量物理学中的量来学习物理学。这些物理量包括长度、时间、质量、温度、压强和电流。
	
	我们通过和一个标准相比较,用它们各自的单位来量度各个
	
	
	
	
	
\end{Paracol}

\subsection{力的三要素}


力是有大小的.我们在初中学过,力的大小可以用测力计来测量.在国际单位制中力的单位是\textbf{牛顿},简称牛,国际符号是N,日常生活和生产中常用的力的单位是千克力,牛顿和千克力的关系是:1千克力$=9.8$牛.

力不但有大小,而且有方向.物体受到的重力是竖直向下的,物体在液体中受到的浮力是竖直向上的,力的方向不同,它的作用效果也不同.用力拉弹簧,弹簧就伸长;用反方向的力压弹簧,弹簧就缩短.作用在运动物体上的力,如果方向与运动方向相同,将加快物体的运动;如果方向与运动方向相反,将阻碍物体的运动.可见,要把一个力完全表达出来,除了说明力的大小外,还要指明力的方向.

\begin{wrapfigure}[6]{r}{6cm}
	\vspace*{-1.5em}
	\centering
	\begin{tikzpicture}[>=stealth, thick]
		\draw (0,0)--(2,0)--(2,1)--(0,1)--(0,0);
		\fill [pattern = north east lines] (-1,-.75) rectangle (3,-.5);
		\draw(-1,-.5)--(3,-.5);
		\draw (.5,-.25) circle (.225);
		\draw (1.5,-.25) circle (.225);
		\draw (.5,-.25) circle (.1);
		\draw (1.5,-.25) circle (.1);
		\draw[->, dashed](3,0.5)--(-2,0.5);
		\draw[very thick](0,.5)--(-2,0.5);
		\foreach \i in {-.4,-.8,-1.2,-1.6,-2}
		{
			\draw (\i,0.6)--(\i,.5);
		}
		\draw (-1,1)--(-1,1.1);
		\draw (-0.6,1)--(-0.6,1.1);
		\draw[-] (-1,1)--(-0.6,1);
		\node at (-0.8,1.35){$\qty{20}{N}$};
	\end{tikzpicture}
	\caption{图中的虚线表示力的作用线}
\end{wrapfigure}
为了直观地说明力的作用,常常用一根带箭头的线段来表示力.线段是按一定比
例(标度)画出的,它的长短表示力的大小,它的指向表示力的方向,箭头或箭尾表示力的作用点,箭头所沿的直线叫做力的作用线.这种表示力的方法,叫做\textbf{力的图示}.图1.1中表示的是作用在小车上的$\qty{100}{N}$的力.

\subsection{力的分类}

我们从初中开始学习物理以来,见过的力的名称已经相当多了.各种力可以用不同的方法来分类.一种是根据力的性质来分类的,如重力、弹力、摩擦力、分子力、电磁力等;另一种是根据力的效果来分类的,如拉力、压力、支持力、动力、阻力等等.拉力、压力、支持力实际上都是弹力,只是效果不同.不论是什么性质的力,只要效果是加快物体的运动,就可以叫它为动力;效果是阻碍物体的运动,就可以叫它为阻力.今后我们还会遇到根据效果来命名的力的名称.

从力的性质来看,力学中经常遇到的有重力、弹力、摩擦力.下面几节就分别介绍这三种力.


\begin{Example}
	测试文字
\end{Example}


\Improve
	\begin{NumberedItem}[3]
			\item test
			\item test
			\item test
			\item test
			\item test
			\item test
			\item test
			\item test
			\item test
			\item test
			\item test
			\item test
			\item test
			\item test
			\item test
			\item test
			\item test
			\item test
			\item test
			\item test
			\item test
			\item test
			\item test
			\item test
			\item test
			\item test
			\item test
			\item test
			\item test
			\item test
			\item test
			\item test
			\item test
			\item test
			\item test
			\item test
	\end{NumberedItem}


\Thinking
	\begin{NumberedItem}
		\item 测试文字
	\end{NumberedItem}


\begin{Definition}
	测试文字
\end{Definition}

\begin{Lemma}
	测试文字
\end{Lemma}

\begin{Theorem}
	测试文字
\end{Theorem}

\begin{Axiom}
	测试文字
\end{Axiom}


\begin{Proposition}
	测试文字
\end{Proposition}


\begin{Corollary}
	测试文字
\end{Corollary}

\begin{Lemma}
	测试文字
\end{Lemma}


\begin{Lemma*}(label)
	[]()
	测试文字
\end{Lemma*}

\pageref{Lem.1.1.1}
\nameref{Lem.1.1.1}
\ref{Lem.1.1.1}


\pageref{Lem.1.1.2}
\nameref{Lem.1.1.2}
\ref{Lem.1.1.2}

\pageref{label}




\Practice{练习题}

\begin{UnnumberedItem}[3]
\item test
\item test
\item test
\item test
\item test
\item test
\item test
\item test
\item test
\end{UnnumberedItem}

\begin{NumberedItem}
	\item 测试文字
	\item test
	\item testqjply
\end{NumberedItem}



\begin{Block}
	\lipsum
\end{Block}
















%\end{Test}
%
%% 第二章
%\include{body/M02.00-zhixianyundong}
%
%% 第三章
%\include{body/M03.00-yundongdinglv}
%
%% 第四章
%\include{body/M04.00-quxianyundong}
%
%% 第五章
%\include{body/M05.00-wanyouyinli}
%
%% 第六章
%\include{body/M06.00-wutidepingheng}
%
%% 第七章
%\include{body/M07.00-jixieneng}
%
%% 第八章
%\include{body/M08.00-dongliang}
%
%% 第九章
%\include{body/M09.00-jixiezhendonghejixiebo}
%
%% 第十章
%\include{body/M010.00-xueshengshiyan}


\backmatter

%\begin{appendix}
%	\include{body/fulu}
%\end{appendix}


\makeback[][barcode.pdf]

%\makespine[1.02cm]

\end{document}



















