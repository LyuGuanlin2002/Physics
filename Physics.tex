\documentclass[themecolor=colorful,openany]{textbook-cn}%,splitbib
\graphicspath{{./figure/}{./figures/}{./image/}{./images/}{./graphics/}{./graphic/}{./pictures/}{./picture/}}%提供多种图片路径
\input{./styles/Cover-cn.tex}
\input{./styles/Logo.tex}
\makeatletter

%%选择题的4个选项,根据选项内容长度自动排版%%
\colorlet{ListColor}{ColorA}
\newlength{\la}
\newlength{\lb}
\newlength{\lc}
\newlength{\ld}
\newlength{\lhalf}
\newlength{\lquarter}
\newlength{\lmax}
\newcommand{\choice}[4]{
\settowidth{\la}{A.~#1~~~}
\settowidth{\lb}{B.~#2~~~}
\settowidth{\lc}{C.~#3~~~}
\settowidth{\ld}{D.~#4~~~}
\ifthenelse{\lengthtest{\la>\lb}}
{\setlength{\lmax}{\la}}{\setlength{\lmax}{\lb}}
\ifthenelse{\lengthtest{\lmax<\lc}}
{\setlength{\lmax}{\lc}}{\relax}
\ifthenelse{\lengthtest{\lmax<\ld}}
{\setlength{\lmax}{\ld}}{\relax}
\setlength{\lhalf}{0.5\linewidth}
\setlength{\lquarter}{0.25\linewidth}
\ifthenelse{\lengthtest{\lmax>\lhalf}}
{\begin{hlist}[pre skip=0pt,item skip=0pt,item offset={1.5em},label={\textbf{\color{ListColor}{\Alpha{hlisti}.}}},pre label={}]1
\hitem #1
\hitem #2
\hitem #3
\hitem #4
\end{hlist}}
{\ifthenelse{\lengthtest{\lmax>\lquarter}}
{\begin{hlist}[pre skip=0pt,item skip=0pt,item offset={1.5em},label={\textbf{\color{ListColor}{\Alpha{hlisti}.}}},pre label={}]2
\hitem #1
\hitem #2
\hitem #3
\hitem #4
\end{hlist}}
{\begin{hlist}[\parskip=0pt,pre skip=0pt,item skip=0pt,item offset={1.5em}, label={\textbf{\color{ListColor}{\Alpha{hlisti}.}}},pre label={}]4
\hitem #1
\hitem #2
\hitem #3
\hitem #4
\end{hlist}}
}}


\makeatother


\usepackage{zhlipsum,lipsum}%乱数假文(测试文字)
\usepackage{physics}%物理符号包
\usepackage{siunitx}%物理符号包
\usetikzlibrary{patterns,3d}

\usepackage{refcount}

%%设置符号表样式%%
\nomensetup{columnsep=2.0em,columns=3}
\makenomenclature

%%设置索引样式%%
\indexsetup{}
\makeindex[name=noun,title=名词索引,columns=3,columnsep=2.0em,options=-s indexstyle.ist]
\makeindex[name=video,title=视频索引,columnsep=2.0em,options=-s indexstyle.ist]

%%设置参考文献样式%%
\bibsetup{intoc}

%%设置书籍信息%%
\series{物理学系列教程}
\version{一}
\title{物理学基础讲义}
\englishtitle{Fundamental of Physics}
\subtitle{经典力学与热力学}
\englishsubtitle{Classical Mechanics \& Thermal Dynamics}
\author{【美】吉尔·沃克\\【美】大卫·哈里德\\【美】罗伯特·瑞斯尼克\\【中】N·A·S·A}
\pressname{中等教育出版社}
\presslogo{gj.pdf}
\coverimage{coverimage.pdf}
\backimage{coverimage.pdf}

\newcommand{\indexnoun}[1]{\index[noun]{#1}}
\newcommand{\indexvideo}[1]{\index[video]{#1}}

\newcommand{\zhindexnoun}[2]{\zhindex[noun]{#1}{#2}}
\newcommand{\zhindexvideo}[2]{\zhindex[video]{#1}{#2}}


%空心字母
\newcommand{\E}{\mathbb{E}}
\renewcommand{\Pr}{\mathbb{P}}
\newcommand{\EP}{\mathbb{E}^{\mathbb{P}}}
\newcommand{\EQ}{\mathbb{E}^{\mathbb{Q}}}
\newcommand{\dif}{\,{\rm d}}
\newcommand{\Var}{{\rm Var}}
\newcommand{\Cov}{{\rm Cov}}

%旧式正切余切函数名 tan cot tanh coth arctan arccot
\newcommand{\sh}{\operatorname{sh}}
\newcommand{\ch}{\operatorname{ch}}
\renewcommand{\th}{\operatorname{th}}
\newcommand{\tg}{\operatorname{tg}}
\newcommand{\ctg}{\operatorname{ctg}}
\newcommand{\tgh}{\operatorname{tgh}}
\newcommand{\ctgh}{\operatorname{ctgh}}
\newcommand{\arctg}{\operatorname{arctg}}
\newcommand{\arcctg}{\operatorname{arcctg}}

%龟壳括号,又称六角括号
\newcommand{\lbr}{\ensuremath{\lbrbrak}}
\newcommand{\rbr}{\ensuremath{\rbrbrak}}
\newcommand{\lhbrak}{\ensuremath{\lbrbrak}}
\newcommand{\rhbrak}{\ensuremath{\rbrbrak}}

%表格对齐设置
\newcolumntype{M}{>{\centering\arraybackslash}X}
\newcolumntype{Z}{>{\raggedright\arraybackslash}X}
\newcolumntype{Y}{>{\raggedleft\arraybackslash}X}

\newcommand{\inputlogo}[1]{\raisebox{-0.5em}{\includegraphics[width=1.5cm]{#1}}~~}
%%
\newcommand{\Hinputlogo}[2]{\href{#2}{\inputlogo{#1}}}


%%%%%%%%%%%%%%%%%%%%%%%%%%%%%%%%%%%%%%%%%%%%%%%%%%%%%%%%%%%%%%%%%%%%%%%%%%%%%%%%%%%
\begin{document}
\MainFont

\makecover
\maketitle

\frontmatter

% 内容提要

% 内容提要
    这套书是沃克、哈里德和瑞斯尼克所著《物理学原理》(Principles of Physics)第10版的中译本。它是一部不仅在美国高校中使用率高,而且在世界许多国家的大学中广泛使用的国际经典教材。
    
    全套书的特点是:突出物理概念,物理原理和定理表述严谨、准确,内容广泛联系最新的科学研究、生产实际以及日常生活。书中除包含传统教科书中经典物理学的全部内容外,第37章到第44章还简要介绍了近代物理学的主要进展,包括相对论、量子物理学,最后一章还提到了夸克和大爆炸。关于引力的第13章,详细讨论了牛顿的引力理论,还简单介绍了等效原理和弯曲时空。书中有许多来自生活和新科技的生动有趣的例子,对提高学生学习物理学的兴趣、明确学习物理学的目的大有裨益。
    
    全套书分上、下两卷,共44章,上卷从第1章到第20章;下卷从第21章到第44章。为了能更好地帮助学生理解和领会物理学的原理和概念,书中每章第一单元的开头都安排有“什么是物理学”的栏目,强调这一章中的主要物理内容。每章中的每单元均设有“学习目标”和“关键概念”的栏目,列出学生学完该单元后必须掌握的物理内容及物理概念。书中的例题分成若干小题,每一小题都详细地交代了解题的关键概念和解题步骤。每章都附有丰富的习题以供教师和学生选择。
    
    本套书的插图内容丰富,表意清楚,制作精美,与正文配合密切。本套书可用作高等学校基础物理课的教材或参考书,同时也是广大教师(包括中学教师)、物理科研人员和物理爱好者十分有价值的参考书。
    
% 序言



\chapter{前言}




%\subsubsection*{我为什么要写这本书}


面对巨大的挑战很有乐趣。这是我对学物理的看法。这想法来自那一天,我曾教的一位名叫莎伦的学生(现已毕业)突然问我:“在我的生活中,这些东西有什么用呢?”当然我立即回答说:“莎伦,在你的生活中每件事都和这有关——这就是物理学。”

她要我举一个例子,我想来想去就是想不出一个合适的。那个晚上我就开始写《物理学的飞行马戏团》(John Wiley \& Sons公司,1975)。这本书是为莎伦而写,也是为我自己而写,因为我知道她的疑问也是我的疑问。我花了六年时间认真钻研了几十本物理学教科书,这些书都是根据最好的教学计划认真地编写出来的,但都缺少了某些东西。物理学是最有趣的学科之一,因为它是关于自然界是怎样运行的,但这些教科书完全没有谈到和真实自然界的任何关系,有趣的东西也都一点没有了。

我已经在《物理学原理》这本书中采纳了从新版的《物理学的飞行马戏团》中挑选出来的许多真实世界物理学的例子。许多材料来自我教的物理学导论课程。在这些课上,我可以从学生的面部表情和直率的评论判断哪些材料和展示有好的效果,而哪些却没有。我的成功和失败记录是形成这本书的基础。这里我要传达的信息和多年前遇到莎伦以来我给我遇到过的每一位学生传达的都同样是:“你们从基本的物理概念终究可以推导出有关真实世界的合理的结论,这种对真实世界的认识就是乐趣之所在。”

我写这本书有好几个目标,但首要的目标是给教师提供一些工具,他们据此可以教学生如何有效地阅读科学资料,懂得基本概念,思考科学问题,并且能够定量地解题。无论对学生还是教师来说,这个过程并不容易。确实,使用这本书的课程或许是学生学过的所有课程中最具挑战性的一门课。然而,它也可能是最值得做的一件事,因为它揭示了所有科学和工程应用赖以实现的自然界的基本机理。

本书第9版的许多使用者(包括教师和学生)给我提出了改进本书的批评意见和建议。这些改进都体现在全书的叙述和习题中。出版商John Wiley \& Sons公司和我把这本书看作不断发展的项目并鼓励使用者提出更多的意见。你们可以把建议、修改意见以及正面或负面的意见送交John Wiley \& Sons或吉尔·沃克(通信地址:克利夫兰州立大学物理系,Cleveland,OH44115 USA);或博客地址:www.flying circus of physics.com)。我们可能无法对所有的建议都做出回应,但我们会尽量保留并研究每一条建议。\footnote{看看我}


\subsubsection*{哪些是新的东西?}
\paragraph{单元和学习目标}“我要从这一节学习到什么?”几十年来最好的学生和最差的学生都问过我这个问题。问题在于,即使是一个善于思考的学生在阅读一个小节时,对是否抓住了要点也可能会感到没有信心。回想起我在用第1版哈里德和瑞斯尼克合著的《物理学》教第一学年的物理学课程时也有同样的感受。

在这一版中为使这个问题缓解一些,我在原来题目的基础上把各章重组成概念的单元,并将各单元的学习目标列出作为每个单元的开始。这些列出的项目是对阅读这一单元应当学到的要点和技巧的简明表述。紧接着每一组列项是对应当学到的关键概念的简明小结。例如,看一看第16章第一单元,学生在这一单元中要面临一大堆的概念和名词。我现在提供了明晰的检索清单,学生可以靠自己的能力把这些概念收集和分类,它的作用就好像飞行员起飞前在跑道上滑行时要通盘核对一遍程序表格一样。

\paragraph{课外作业习题和学习目标之间的关系}
在WileyPLUS中,每一章后面的每一个问题和习题都联系着学习目标,要回答(通常不用说出来)这样的问题:“我为什么要做这个习题?我应该从它学到什么?”通过明确一个习题的目的,我相信学生会用不同的语言但却相同的关键概念把学习目标更好地转移到其他的习题上。这种转移有助于克服常常遇到的困难,就是学生学会了解某一特殊的习题,但却不会把它的关键概念用于另一种条件下的问题。

\paragraph{重写某几章}我的学生对关键的几章和另外几章中的一些方面不断地提出建议,所以在这一版中我重写了许多内容。例如,我重新构思了关于高斯定律和电势这两章,因为原先的这两章被证明对我的学生来说太困难了。现在的表述更加流畅,并且关键的要点表述更加直截了当。在有关量子物理的几章里,我扩展了薛定谔方程的范围,包括物质波在阶跃势上的反射。遵照一些教师的要求,我将玻尔原子的讨论和氢原子的薛定谔解分开,这样就可以绕过对玻尔工作的历史说明。还有,现在有了关于普朗克的黑体辐射的单元。

\paragraph{新的例题}16个新的例题已经被增加到各章中,这是为了突出我的学生们感到困难的一些领域。

\paragraph{可视图解}在WileyPLUS中可以得到这本教材的电子版,这是罗格斯大学(Rutgers University)的戴维·梅洛(David Maiullo)制作的教材中大约30幅照片和插图的视频。物理学中大部分是研究运动的事物,视频常常可以比静态的照片和图片提供更佳的描述。

\paragraph{在线辅助}WileyPLUS不仅仅是在线评分的程序,实际上它还是生动的学习中心,配有许多不同的学习辅助材料,包括实时的解题指导、鼓励学生的嵌入式阅读测验、动画、几百道例题、大量的模拟和演示以及1500个以上的视频,内容从数学复习到与例题有关的微型课程。每学期都会增加更多的学习辅助材料。在《物理学原理》第10版中,一些运动的照片被转换成视频,这样可以将运动变慢以进行分析。

这几千个学习辅助材料可以全天候得到,并且可以随意重复使用。这样,如果一个学生,譬如说在半夜2点40分(这好像是做物理课外作业的最佳时间)被一个课外作业题难住,点击鼠标就可以得到合适的、对他有帮助的资料。

\subsubsection*{学习工具}
当我用第1版哈里德和瑞斯尼克合著的《物理学》学习第一年物理学时,我通过反复阅读才得以理解每一章。现今,我们更好地了解到,学生有广泛多样的学习风格,所以我制作了多样的学习工具,这些都体现在了这一版的书和在线的WileyPLUS中。

\paragraph{动画}每一章都有一些关键的图用动画表现。在这本书中,这些图用旋涡符号来标记。在WileyPLUS的线上章节中,点击鼠标动画就开始了。我选择其中有丰富信息的一些图做成动画,故学生可以看到的不仅仅是那些印刷在书页上的插图,而是活生生的物理学,并且用几分钟时间就可以播放完。这不但使物理学鲜活起来,而且动画可以根据学生的需要多次重复播放。

\paragraph{视频}我已经制作了1500多个教学视频,每学期还要增加,学生可以听我在解释、指导、讲解例题或总结的时候在屏幕上看着我画图或打字,十分像他们在我的办公室里坐在我旁边看我在草稿本上推算某些东西时的经历一样。教师的讲课或个别指导总是最有价值的学习方式,而我们视频是全天候都可以得到的,并且可以无数次地重复使用。

\begin{UnnumberedItem}
\item 一些章节中某些主题的视频辅导。我选择学生感到最困难、最伤脑筋的一些主题。

\item 高中数学的视频复习,像基本的代数运算、三角函数和联立方程。

\item 数学的视频介绍,像矢量运算,这对学生来说是新的知识。

\item 教材各章中每个例题的视频图像描述。我的意图是从关键概念出发学习物理学而不是只抓住公式。然而,我还是会演示怎样解读例题,就是说怎样读懂技术资料,学会解题的步骤,这些也可以用到其他类型的习题上。

\item 每一章后面20\% 的习题的视频求解。学生是不是能够看到这些解以及什么时候才能得到答案是由教师控制的。例如,可以在课外作业截止期限以后或者小测验以后得到。每一个解答不是简单的对号入座的处方。我建立了从基本概念和推理的第一步开始到最后的答案的解题方法。学生不仅仅是学习解答一道特定的习题,而是要学会处理任何问题,甚至要有处理这些问题所需要的物理学的勇气。

\item 怎样从曲线图读出数据的视频例子(并不是在没有理解物理意义的情况下就去简单地读取数字)。
\end{UnnumberedItem}




\paragraph{解题助手}我已经为WileyPLUS编写了大量的资料,这些是为帮助学生提高解题能力而设计的。

\begin{UnnumberedItem}
\item 本书中的每道例题的阅读及视频版本都可以在线上得到。

\item 几百道附加的例题。这些都是独一无二的资料,但(可由教师自己选定)它们也连接着超出课外作业范围的例题。所以,如果一道课外作业题是处理,比如说是作用于斜面上的木块的力,那么这里也提供了有关例题的连接。不过,这种例题和课外作业并不完全一样,它并不提供一个只要复制而不用理解的解答。

\item 每一章后面的课外作业中的15\% 都可在GO Tutorials栏目中找出求解步骤。我引导学生做课外作业要经过多个步骤,从关键概念开始,有时给出错误的答案并做出提示。然而,我会故意把(得到最终答案的)最后一步留给学生。这样,他们最后要自己负责做完习题。某些在线教学系统有意给出错误答案让学生落入陷阱,这会使学生产生很大的困惑,而我的GO Tutorials并不是陷阱,学生在解题过程中的每一步都可以回到主要的问题上来。

\item 每一章后面课外作业的每一道题的提示都可以(在教师的指导下)得到。我编写的这些材料是关于主要概念和解题一般步骤的具体提示,而不是只提供答案而无须理解的诀窍。
\end{UnnumberedItem}

\paragraph{评价资料}

\begin{UnnumberedItem}
\item 在线上的每一节都可找到相应的阅读问题。我编写这些材料并不是要让他们进行分析或深入的理解,只是为了测试一下学生是不是读过这一节。当学生打开某一节时,从题库中随机选择的阅读问题就会出现在该节最后的空白处。教师可以自行决定这个问题是作为打分数的根据呢,还是仅仅作为学生的练习。在线上的每一节都可找到相应的阅读问题。我编写这些材料并不是要让他们进行分析或深入的理解,只是为了测试一下学生是不是读过这一节。当学生打开某一节时,从题库中随机选择的阅读问题就会出现在该节最后的空白处。教师可以自行决定这个问题是作为打分数的根据呢,还是仅仅作为学生的练习。

\item 在大多数小节中设置有检查点。这些检查点要求用这一节中的物理原理做分析和判断。所有检查点的答案都在书的最后。

\item 本书中每一章后面的大多数习题(和更多其他的习题)在WileyPLUS中都可以找到。教师可以在线上指定课外作业,并依据网上提交的答案打分。例如,教师规定交作业的截止日期和允许一个学生对一个答案可以尝试多少次。教师也可以控制每一道课外习题能得到哪些学习帮助(如果有的话)。这种连接包括提示、例题、章内的阅读材料、视频辅导、视频教学复习,甚至还包括视频解题(这可以在课外作业截止日期后给学生)。

\item 符号标记的习题。这种需要得到代数式答案的习题在每章中都有。
\end{UnnumberedItem}




\subsubsection*{教师用的补充资料}

\paragraph{教师用解题手册}由Lawrence Livermore国家实验室的Sen-Ben Liao编著。这本手册提供每章后面所有习题的解题步骤,它有MS Word和PDF两种格式。

\paragraph{教师伴侣网址}http://www.wiley.com/college/halliday

\paragraph{教师手册}这份资料概括了每一章中最重要的论题的讲课要点、演示实验、实验室和计算机项目、电影和视频资料、所有习题的答案和检查点以及与以前版本中习题相关的指导,也包含了学生可以得到解答的所有习题的完整目录。

\paragraph{讲课用Power Point幻灯片}这些Power Point幻灯片可用作对教师有帮助的起动包,它概括了这本教材中的关键概念和相关的图表及方程式。

\paragraph{Wiley物理学模拟}由Boston University的Andrew Duffy和Vernier Software的John Gastineau制作。这是50个相互作用的模拟(Java应用程序),可以用作课堂演示。

\paragraph{Wiley物理学演示}由Rutgers University的David Maiullo制作。这是80个标准物理学演示的数字视频的集合。它们可以在课堂上演示或从WileyPLUS中得到。另有与选择题相配套的教师指导。

\paragraph{试题库}第10版的试题库已被Northern Illinois University的Suzanne Willis全部检查过,试题库包含2200多道多项选择题。这些题目在计算机试题库中也可找到,这个计算机试题库提供了完整的编辑功能,可以帮助你按自己的要求选择测验题(IBM和Macintosh版本都可以得到)。

\paragraph{教材中所有的图表}适用于课堂投影或印刷。


\paragraph{线上课外作业和小测验}除了WileyPLUS和《物理学原理》第10版外,也支持WebAssign PLUS和LON-CAPA,这些程序也提供教师在线上布置课外作业和小测验以及评分的功能。WebAssignPLUS也给学生提供这本教材的线上版本。

\subsubsection*{学生用的补充资料}
\paragraph{学生伴侣}网址http://www.wiley.com/college/hallidy,这是专门为《物理学原理》第10版制作并为进一步帮助学生学习物理学而设计的,它包含每一章后面的部分习题的解答、模拟练习,以及怎样最好地应用可编程计算器的技巧。

\paragraph{互动学习软件}这个软件指导学生如何求解200道各章后面的习题。求解过程是互动的,有适当的反馈并可得到防止最常见错误的具体指导。




















\shorttableofcontents
\tableofcontents

%\listoffigures
%
\mainmatter

%\partimage{
%\begin{tikzpicture}[scale=0.8]
%	%============ 参数配置 ============%
%	\def\archWidth{14}        % 支架总宽度
%	\def\archHeight{8}        % 支架立柱高度
%	\def\archRadius{8mm}      % 顶部圆角半径
%	\def\lineSep{12}          % 悬挂线间距
%	\def\numBalls{5}          % 小球数量
%	\def\hangLength{6}        % 悬挂线长度
%	\def\swingAngle{-10}      % 左摆角度
%	\def\baseHeight{1.5}        % 底座高度
%	\def\baseDownWidth{17}    % 底座下底宽度
%	\def\baseUpWidth{16}      % 底座上底宽度
%	
%	%============ 自动计算参数 ============%
%	\pgfmathsetmacro\spacing{\lineSep/(\numBalls+1)}  % 小球间距
%	\pgfmathsetmacro\ballRadius{\spacing/2}           % 小球半径
%	\pgfmathsetmacro\startX{-\lineSep/2 + \spacing}   % 起始坐标
%	
%	%============ 倒U型支架 ============%
%	\draw[white,line width=3.0mm,rounded corners=\archRadius] 
%	(-\archWidth/2,0) -- (-\archWidth/2,\archHeight) 
%	-- (\archWidth/2,\archHeight) -- (\archWidth/2,0);
%	
%	%============ 正梯形底座 ============%
%	\draw[white,line width=1.5mm,fill=white,rounded corners] 
%	(-\baseUpWidth/2,0) -- 
%	(\baseUpWidth/2,0) -- 
%	(\baseDownWidth/2,-\baseHeight) -- 
%	(-\baseDownWidth/2,-\baseHeight) -- cycle;
%	
%	%============ 悬挂系统 ============%
%	\foreach \i in {0,...,4} {
%		\pgfmathsetmacro\xPos{\startX + \i*\spacing}
%		
%		% 左侧摆动球
%		\ifnum\i=0
%		\draw[white,line width=1.5mm,rotate around={\swingAngle:(\xPos,\archHeight)}] 
%		(\xPos,\archHeight) -- ++(0,-\hangLength) coordinate (ballL);
%		\draw[white,fill=white] (ballL) circle (\ballRadius);
%		\fi
%		
%		% 右侧静止球
%		\ifnum\i>0
%		\pgfmathsetmacro\mirrorX{-\xPos}
%		\draw[white,line width=1.5mm] (\xPos,\archHeight) -- (\xPos,\archHeight-\hangLength) coordinate (ballR);
%		\draw[white,fill=white] (ballR) circle (\ballRadius);
%		% 左上角高光
%		\coordinate(B)at(\xPos,\archHeight-\hangLength+\ballRadius*0.6);
%		\draw[ColorB!15,line width=1.0mm] (B) arc (90:180:\ballRadius*0.6);
%		\fi
%	}
%	\node[inner sep=15mm,anchor=north east,, transparent]at(-\baseDownWidth/2,-\baseHeight){};
%\end{tikzpicture}
%
%}



\partintro{\lipsum[2]}
\part{力学}

%% 第一章
\chaptersaying{文王拘而演周易,仲尼厄而作春秋。}
\chapter{测量}





\section{测量物体,包含长度测量}

\begin{Point*}
	学习完这一单元,你应当能够: 识别国际单位制(SI)中的基本量; 正确称呼国际单位制中最常用的词头; 利用链环变换法改变单位(这里是对于长度、面积和体积); 说明米是如何用真空中光速来定义的。
\end{Point*}


\begin{Case*}
物理学是以物理量的测量为基础的。一些物理量被选作基本物理量; 每一种基本物理量都已经通过一种标准来定义并给定测量的单位。另一些物理量是用这些基本物理量和它们的标准及单位来定义。

本书主要用的单位制是国际单位制(SI)。基本物理量的标准必须都是可以得到并且是不变的,这些基本量的标准都已经通过国际协议确立了。这些标准用在所有的物理测量中,基本量和从这些基本量导出的量都是如此。

单位的变换可以通过链环变换法来完成,在这种变换法中,将原来的数据连续乘以统一协调的变换因子,单位要像代数量一样处理,直到留下所要求的单位。

米定义为在精确规定的时间间隔内光通过的路程。
\end{Case*}



\begin{Paracol}
	\subsection{什么是物理学?}
	
	科学和工程学建立在测量和比较的基础上。因此,我们就需要关于如何测量和比较物体的规则,并且我们还需要通过实验来建立这种测量和比较的单位。物理学(以及工程学)的一个目的就是设计和实施这些实验。
	
	例如,物理学家们尽力开发极其准确的时钟从而使任何时间或时间间隔都可以精确地确定和比较。你们可能要问,这样的准确度是否真正需要,或者是否值得花这么大的精力。这里给出一个是否值得的例子:如果没有极其准确的时钟的话,那么现在全世界范围的航行必不可少的全球定位系统(GPS)就会变得毫无意义。
	
	\subsection{测量物体}
	
	我们通过学习怎样测量物理学中的量来学习物理学。这些物理量包括长度、时间、质量、温度、压强和电流。
	
	我们通过和一个标准相比较,用它们各自的单位来量度各个
	
	
	
	
	
\end{Paracol}

\subsection{力的三要素}


力是有大小的.我们在初中学过,力的大小可以用测力计来测量.在国际单位制中力的单位是\textbf{牛顿},简称牛,国际符号是N,日常生活和生产中常用的力的单位是千克力,牛顿和千克力的关系是:1千克力$=9.8$牛.

力不但有大小,而且有方向.物体受到的重力是竖直向下的,物体在液体中受到的浮力是竖直向上的,力的方向不同,它的作用效果也不同.用力拉弹簧,弹簧就伸长;用反方向的力压弹簧,弹簧就缩短.作用在运动物体上的力,如果方向与运动方向相同,将加快物体的运动;如果方向与运动方向相反,将阻碍物体的运动.可见,要把一个力完全表达出来,除了说明力的大小外,还要指明力的方向.

\begin{wrapfigure}[6]{r}{6cm}
	\vspace*{-1.5em}
	\centering
	\begin{tikzpicture}[>=stealth, thick]
		\draw (0,0)--(2,0)--(2,1)--(0,1)--(0,0);
		\fill [pattern = north east lines] (-1,-.75) rectangle (3,-.5);
		\draw(-1,-.5)--(3,-.5);
		\draw (.5,-.25) circle (.225);
		\draw (1.5,-.25) circle (.225);
		\draw (.5,-.25) circle (.1);
		\draw (1.5,-.25) circle (.1);
		\draw[->, dashed](3,0.5)--(-2,0.5);
		\draw[very thick](0,.5)--(-2,0.5);
		\foreach \i in {-.4,-.8,-1.2,-1.6,-2}
		{
			\draw (\i,0.6)--(\i,.5);
		}
		\draw (-1,1)--(-1,1.1);
		\draw (-0.6,1)--(-0.6,1.1);
		\draw[-] (-1,1)--(-0.6,1);
		\node at (-0.8,1.35){$\qty{20}{N}$};
	\end{tikzpicture}
	\caption{图中的虚线表示力的作用线}
\end{wrapfigure}
为了直观地说明力的作用,常常用一根带箭头的线段来表示力.线段是按一定比
例(标度)画出的,它的长短表示力的大小,它的指向表示力的方向,箭头或箭尾表示力的作用点,箭头所沿的直线叫做力的作用线.这种表示力的方法,叫做\textbf{力的图示}.图1.1中表示的是作用在小车上的$\qty{100}{N}$的力.

\subsection{力的分类}

我们从初中开始学习物理以来,见过的力的名称已经相当多了.各种力可以用不同的方法来分类.一种是根据力的性质来分类的,如重力、弹力、摩擦力、分子力、电磁力等;另一种是根据力的效果来分类的,如拉力、压力、支持力、动力、阻力等等.拉力、压力、支持力实际上都是弹力,只是效果不同.不论是什么性质的力,只要效果是加快物体的运动,就可以叫它为动力;效果是阻碍物体的运动,就可以叫它为阻力.今后我们还会遇到根据效果来命名的力的名称.

从力的性质来看,力学中经常遇到的有重力、弹力、摩擦力.下面几节就分别介绍这三种力.


\begin{Example}
	测试文字
\end{Example}


\Improve
	\begin{NumberedItem}[3]
			\item test
			\item test
			\item test
			\item test
			\item test
			\item test
			\item test
			\item test
			\item test
			\item test
			\item test
			\item test
			\item test
			\item test
			\item test
			\item test
			\item test
			\item test
			\item test
			\item test
			\item test
			\item test
			\item test
			\item test
			\item test
			\item test
			\item test
			\item test
			\item test
			\item test
			\item test
			\item test
			\item test
			\item test
			\item test
			\item test
	\end{NumberedItem}


\Thinking
	\begin{NumberedItem}
		\item 测试文字
	\end{NumberedItem}


\begin{Definition}
	测试文字
\end{Definition}

\begin{Lemma}
	测试文字
\end{Lemma}

\begin{Theorem}
	测试文字
\end{Theorem}

\begin{Axiom}
	测试文字
\end{Axiom}


\begin{Proposition}
	测试文字
\end{Proposition}


\begin{Corollary}
	测试文字
\end{Corollary}

\begin{Lemma}
	测试文字
\end{Lemma}


\begin{Lemma*}(label)
	[]()
	测试文字
\end{Lemma*}

\pageref{Lem.1.1.1}
\nameref{Lem.1.1.1}
\ref{Lem.1.1.1}


\pageref{Lem.1.1.2}
\nameref{Lem.1.1.2}
\ref{Lem.1.1.2}

\pageref{label}




\Practice{练习题}

\begin{UnnumberedItem}[3]
\item test
\item test
\item test
\item test
\item test
\item test
\item test
\item test
\item test
\end{UnnumberedItem}

\begin{NumberedItem}
	\item 测试文字
	\item test
	\item testqjply
\end{NumberedItem}



\begin{Block}
	\lipsum
\end{Block}

















%% 第二章
%\include{main/M02.00-zhixianyundong}

%% 第三章
%\include{main/M03.00-yundongdinglv}

%% 第四章
%\include{body/M04.00-quxianyundong}
%
%% 第五章
%\include{body/M05.00-wanyouyinli}
%
%% 第六章
%\include{body/M06.00-wutidepingheng}
%
%% 第七章
%\include{body/M07.00-jixieneng}
%
%% 第八章
%\include{body/M08.00-dongliang}
%
%% 第九章
%\include{body/M09.00-jixiezhendonghejixiebo}
%
%% 第十章
%\include{body/M010.00-xueshengshiyan}


\backmatter

%\begin{appendix}
%	\include{body/fulu}
%\end{appendix}

\chapter{后记}

\lipsum

\backintro{\lipsum[2]}
\barcode{barcode.pdf}
\makeback

\makespine[1.02cm]

\end{document}



















