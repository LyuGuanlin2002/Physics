


\chapter{前言}




%\subsubsection*{我为什么要写这本书}


面对巨大的挑战很有乐趣。这是我对学物理的看法。这想法来自那一天,我曾教的一位名叫莎伦的学生(现已毕业)突然问我:“在我的生活中,这些东西有什么用呢?”当然我立即回答说:“莎伦,在你的生活中每件事都和这有关——这就是物理学。”

她要我举一个例子,我想来想去就是想不出一个合适的。那个晚上我就开始写《物理学的飞行马戏团》(John Wiley \& Sons公司,1975)。这本书是为莎伦而写,也是为我自己而写,因为我知道她的疑问也是我的疑问。我花了六年时间认真钻研了几十本物理学教科书,这些书都是根据最好的教学计划认真地编写出来的,但都缺少了某些东西。物理学是最有趣的学科之一,因为它是关于自然界是怎样运行的,但这些教科书完全没有谈到和真实自然界的任何关系,有趣的东西也都一点没有了。

我已经在《物理学原理》这本书中采纳了从新版的《物理学的飞行马戏团》中挑选出来的许多真实世界物理学的例子。许多材料来自我教的物理学导论课程。在这些课上,我可以从学生的面部表情和直率的评论判断哪些材料和展示有好的效果,而哪些却没有。我的成功和失败记录是形成这本书的基础。这里我要传达的信息和多年前遇到莎伦以来我给我遇到过的每一位学生传达的都同样是:“你们从基本的物理概念终究可以推导出有关真实世界的合理的结论,这种对真实世界的认识就是乐趣之所在。”

我写这本书有好几个目标,但首要的目标是给教师提供一些工具,他们据此可以教学生如何有效地阅读科学资料,懂得基本概念,思考科学问题,并且能够定量地解题。无论对学生还是教师来说,这个过程并不容易。确实,使用这本书的课程或许是学生学过的所有课程中最具挑战性的一门课。然而,它也可能是最值得做的一件事,因为它揭示了所有科学和工程应用赖以实现的自然界的基本机理。

本书第9版的许多使用者(包括教师和学生)给我提出了改进本书的批评意见和建议。这些改进都体现在全书的叙述和习题中。出版商John Wiley \& Sons公司和我把这本书看作不断发展的项目并鼓励使用者提出更多的意见。你们可以把建议、修改意见以及正面或负面的意见送交John Wiley \& Sons或吉尔·沃克(通信地址:克利夫兰州立大学物理系,Cleveland,OH44115 USA);或博客地址:www.flying circus of physics.com)。我们可能无法对所有的建议都做出回应,但我们会尽量保留并研究每一条建议。\footnote{看看我}


\subsubsection*{哪些是新的东西?}
\paragraph{单元和学习目标}“我要从这一节学习到什么?”几十年来最好的学生和最差的学生都问过我这个问题。问题在于,即使是一个善于思考的学生在阅读一个小节时,对是否抓住了要点也可能会感到没有信心。回想起我在用第1版哈里德和瑞斯尼克合著的《物理学》教第一学年的物理学课程时也有同样的感受。

在这一版中为使这个问题缓解一些,我在原来题目的基础上把各章重组成概念的单元,并将各单元的学习目标列出作为每个单元的开始。这些列出的项目是对阅读这一单元应当学到的要点和技巧的简明表述。紧接着每一组列项是对应当学到的关键概念的简明小结。例如,看一看第16章第一单元,学生在这一单元中要面临一大堆的概念和名词。我现在提供了明晰的检索清单,学生可以靠自己的能力把这些概念收集和分类,它的作用就好像飞行员起飞前在跑道上滑行时要通盘核对一遍程序表格一样。

\paragraph{课外作业习题和学习目标之间的关系}
在WileyPLUS中,每一章后面的每一个问题和习题都联系着学习目标,要回答(通常不用说出来)这样的问题:“我为什么要做这个习题?我应该从它学到什么?”通过明确一个习题的目的,我相信学生会用不同的语言但却相同的关键概念把学习目标更好地转移到其他的习题上。这种转移有助于克服常常遇到的困难,就是学生学会了解某一特殊的习题,但却不会把它的关键概念用于另一种条件下的问题。

\paragraph{重写某几章}我的学生对关键的几章和另外几章中的一些方面不断地提出建议,所以在这一版中我重写了许多内容。例如,我重新构思了关于高斯定律和电势这两章,因为原先的这两章被证明对我的学生来说太困难了。现在的表述更加流畅,并且关键的要点表述更加直截了当。在有关量子物理的几章里,我扩展了薛定谔方程的范围,包括物质波在阶跃势上的反射。遵照一些教师的要求,我将玻尔原子的讨论和氢原子的薛定谔解分开,这样就可以绕过对玻尔工作的历史说明。还有,现在有了关于普朗克的黑体辐射的单元。

\paragraph{新的例题}16个新的例题已经被增加到各章中,这是为了突出我的学生们感到困难的一些领域。

\paragraph{可视图解}在WileyPLUS中可以得到这本教材的电子版,这是罗格斯大学(Rutgers University)的戴维·梅洛(David Maiullo)制作的教材中大约30幅照片和插图的视频。物理学中大部分是研究运动的事物,视频常常可以比静态的照片和图片提供更佳的描述。

\paragraph{在线辅助}WileyPLUS不仅仅是在线评分的程序,实际上它还是生动的学习中心,配有许多不同的学习辅助材料,包括实时的解题指导、鼓励学生的嵌入式阅读测验、动画、几百道例题、大量的模拟和演示以及1500个以上的视频,内容从数学复习到与例题有关的微型课程。每学期都会增加更多的学习辅助材料。在《物理学原理》第10版中,一些运动的照片被转换成视频,这样可以将运动变慢以进行分析。

这几千个学习辅助材料可以全天候得到,并且可以随意重复使用。这样,如果一个学生,譬如说在半夜2点40分(这好像是做物理课外作业的最佳时间)被一个课外作业题难住,点击鼠标就可以得到合适的、对他有帮助的资料。

\subsubsection*{学习工具}
当我用第1版哈里德和瑞斯尼克合著的《物理学》学习第一年物理学时,我通过反复阅读才得以理解每一章。现今,我们更好地了解到,学生有广泛多样的学习风格,所以我制作了多样的学习工具,这些都体现在了这一版的书和在线的WileyPLUS中。

\paragraph{动画}每一章都有一些关键的图用动画表现。在这本书中,这些图用旋涡符号来标记。在WileyPLUS的线上章节中,点击鼠标动画就开始了。我选择其中有丰富信息的一些图做成动画,故学生可以看到的不仅仅是那些印刷在书页上的插图,而是活生生的物理学,并且用几分钟时间就可以播放完。这不但使物理学鲜活起来,而且动画可以根据学生的需要多次重复播放。

\paragraph{视频}我已经制作了1500多个教学视频,每学期还要增加,学生可以听我在解释、指导、讲解例题或总结的时候在屏幕上看着我画图或打字,十分像他们在我的办公室里坐在我旁边看我在草稿本上推算某些东西时的经历一样。教师的讲课或个别指导总是最有价值的学习方式,而我们视频是全天候都可以得到的,并且可以无数次地重复使用。

\begin{UnnumberedItem}
\item 一些章节中某些主题的视频辅导。我选择学生感到最困难、最伤脑筋的一些主题。

\item 高中数学的视频复习,像基本的代数运算、三角函数和联立方程。

\item 数学的视频介绍,像矢量运算,这对学生来说是新的知识。

\item 教材各章中每个例题的视频图像描述。我的意图是从关键概念出发学习物理学而不是只抓住公式。然而,我还是会演示怎样解读例题,就是说怎样读懂技术资料,学会解题的步骤,这些也可以用到其他类型的习题上。

\item 每一章后面20\% 的习题的视频求解。学生是不是能够看到这些解以及什么时候才能得到答案是由教师控制的。例如,可以在课外作业截止期限以后或者小测验以后得到。每一个解答不是简单的对号入座的处方。我建立了从基本概念和推理的第一步开始到最后的答案的解题方法。学生不仅仅是学习解答一道特定的习题,而是要学会处理任何问题,甚至要有处理这些问题所需要的物理学的勇气。

\item 怎样从曲线图读出数据的视频例子(并不是在没有理解物理意义的情况下就去简单地读取数字)。
\end{UnnumberedItem}




\paragraph{解题助手}我已经为WileyPLUS编写了大量的资料,这些是为帮助学生提高解题能力而设计的。

\begin{UnnumberedItem}
\item 本书中的每道例题的阅读及视频版本都可以在线上得到。

\item 几百道附加的例题。这些都是独一无二的资料,但(可由教师自己选定)它们也连接着超出课外作业范围的例题。所以,如果一道课外作业题是处理,比如说是作用于斜面上的木块的力,那么这里也提供了有关例题的连接。不过,这种例题和课外作业并不完全一样,它并不提供一个只要复制而不用理解的解答。

\item 每一章后面的课外作业中的15\% 都可在GO Tutorials栏目中找出求解步骤。我引导学生做课外作业要经过多个步骤,从关键概念开始,有时给出错误的答案并做出提示。然而,我会故意把(得到最终答案的)最后一步留给学生。这样,他们最后要自己负责做完习题。某些在线教学系统有意给出错误答案让学生落入陷阱,这会使学生产生很大的困惑,而我的GO Tutorials并不是陷阱,学生在解题过程中的每一步都可以回到主要的问题上来。

\item 每一章后面课外作业的每一道题的提示都可以(在教师的指导下)得到。我编写的这些材料是关于主要概念和解题一般步骤的具体提示,而不是只提供答案而无须理解的诀窍。
\end{UnnumberedItem}

\paragraph{评价资料}

\begin{UnnumberedItem}
\item 在线上的每一节都可找到相应的阅读问题。我编写这些材料并不是要让他们进行分析或深入的理解,只是为了测试一下学生是不是读过这一节。当学生打开某一节时,从题库中随机选择的阅读问题就会出现在该节最后的空白处。教师可以自行决定这个问题是作为打分数的根据呢,还是仅仅作为学生的练习。在线上的每一节都可找到相应的阅读问题。我编写这些材料并不是要让他们进行分析或深入的理解,只是为了测试一下学生是不是读过这一节。当学生打开某一节时,从题库中随机选择的阅读问题就会出现在该节最后的空白处。教师可以自行决定这个问题是作为打分数的根据呢,还是仅仅作为学生的练习。

\item 在大多数小节中设置有检查点。这些检查点要求用这一节中的物理原理做分析和判断。所有检查点的答案都在书的最后。

\item 本书中每一章后面的大多数习题(和更多其他的习题)在WileyPLUS中都可以找到。教师可以在线上指定课外作业,并依据网上提交的答案打分。例如,教师规定交作业的截止日期和允许一个学生对一个答案可以尝试多少次。教师也可以控制每一道课外习题能得到哪些学习帮助(如果有的话)。这种连接包括提示、例题、章内的阅读材料、视频辅导、视频教学复习,甚至还包括视频解题(这可以在课外作业截止日期后给学生)。

\item 符号标记的习题。这种需要得到代数式答案的习题在每章中都有。
\end{UnnumberedItem}




\subsubsection*{教师用的补充资料}

\paragraph{教师用解题手册}由Lawrence Livermore国家实验室的Sen-Ben Liao编著。这本手册提供每章后面所有习题的解题步骤,它有MS Word和PDF两种格式。

\paragraph{教师伴侣网址}http://www.wiley.com/college/halliday

\paragraph{教师手册}这份资料概括了每一章中最重要的论题的讲课要点、演示实验、实验室和计算机项目、电影和视频资料、所有习题的答案和检查点以及与以前版本中习题相关的指导,也包含了学生可以得到解答的所有习题的完整目录。

\paragraph{讲课用Power Point幻灯片}这些Power Point幻灯片可用作对教师有帮助的起动包,它概括了这本教材中的关键概念和相关的图表及方程式。

\paragraph{Wiley物理学模拟}由Boston University的Andrew Duffy和Vernier Software的John Gastineau制作。这是50个相互作用的模拟(Java应用程序),可以用作课堂演示。

\paragraph{Wiley物理学演示}由Rutgers University的David Maiullo制作。这是80个标准物理学演示的数字视频的集合。它们可以在课堂上演示或从WileyPLUS中得到。另有与选择题相配套的教师指导。

\paragraph{试题库}第10版的试题库已被Northern Illinois University的Suzanne Willis全部检查过,试题库包含2200多道多项选择题。这些题目在计算机试题库中也可找到,这个计算机试题库提供了完整的编辑功能,可以帮助你按自己的要求选择测验题(IBM和Macintosh版本都可以得到)。

\paragraph{教材中所有的图表}适用于课堂投影或印刷。


\paragraph{线上课外作业和小测验}除了WileyPLUS和《物理学原理》第10版外,也支持WebAssign PLUS和LON-CAPA,这些程序也提供教师在线上布置课外作业和小测验以及评分的功能。WebAssignPLUS也给学生提供这本教材的线上版本。

\subsubsection*{学生用的补充资料}
\paragraph{学生伴侣}网址http://www.wiley.com/college/hallidy,这是专门为《物理学原理》第10版制作并为进一步帮助学生学习物理学而设计的,它包含每一章后面的部分习题的解答、模拟练习,以及怎样最好地应用可编程计算器的技巧。

\paragraph{互动学习软件}这个软件指导学生如何求解200道各章后面的习题。求解过程是互动的,有适当的反馈并可得到防止最常见错误的具体指导。


















